\bibliographystyle{plain}
\usetheme{Dresden} %主题
\usepackage{setspace}
\usepackage{graphicx}
\usepackage{subfigure}
\usepackage{times}
\usepackage{amsmath}
\usepackage{url}
\usepackage{hyperref}
\usepackage{biblatex}
\addbibresource{mybib.bib}

% 超链接设置
\hypersetup{colorlinks = true, linkcolor = blue, anchorcolor = blue, citecolor = red}


% 设置行间距
%\renewcommand{\baselinestretch}{1.0}

\newtheorem{remark}{Remark}[section]


\mode<presentation>
{
 % 设置背景
 \setbeamertemplate{background canvas}[vertical shading][bottom=red!10,top=blue!10]
 
 % 设置block的特征
 \setbeamertemplate{blocks}[rounded][shadow=true]
 
 % 设置主题
 \usetheme{Warsaw}
 
 % 设置footline显示页码,但是由于warsaw本身已经定义了一个footline,所以这个定义就会覆盖warsaw的定义。另一方面说明,slides的每一部分都是可以自己定制的。
 % \setbeamertemplate{footline}[frame number]
 
% 添加页码代码,谷歌找到的。
\expandafter\def\expandafter\insertshorttitle\expandafter{%
    \insertshorttitle\hfill%
    \insertframenumber\,/\,\inserttotalframenumber}
 
 
 % 设置覆盖的效果,透明
 \setbeamercovered{transparent}
 
 \usefonttheme[onlysmall]{structurebold}
 
 % 设置数学公式的字体
 \usefonttheme[onlymath]{serif}
}